\section{Introduction}
	\subsection{Background}
		\hspace{10mm}Blockchain technology has been a disruptive force in this world. Originally, that disruption was felt in the financial sector with Bitcoin. This relegation to the financial sector existed because the Bitcoin scripting language Script was intentionally non-Turing complete; perfect for its Use Case as a cryptocurrency, but otherwise significantly under-powered for general applications. However, it wasn't long until the next level came in the form of Ethereum which introduced a Turing-complete scripting language along side its cryptocurrency offering. This opened up the world to the possibility of decentralized applications (dApps). While those platforms were necessarily public and open, others still thought about other Use Cases that would benefit from the use of this type of distributed technology -- ledgers in the form of cryptographically connected blocks, decentralized node consensus, scripting to build applications on these platforms, etc. -- in a private and permissioned space. One of those offerings was a system, itself defined and built by a consortium of organizations (from Intel to IBM), maintained by The Linux Foundation called The Hyperledger Project.\\
	
		\hspace{10mm}FabSec (short for Fabric Security) is an exploration in the potential of using Hyperledger Fabric's Distributed Ledger Technology\footnote{A note about terminology: You'll see a few terms being used here such as Consortium Blockchain and Distributed Ledger. For the purposes of this project, these are interchangeable. This holds as a Blockchain system at its base is nothing more than a Distributed Ledger of transactions. It's just as the public Blockchain sphere increases, that term generally refers to the Bitcoin, Ethereum, et al menagerie of systems. Another example seen later of this is Chaincode vs Smart Contracts.} as a dedicated security network. Hyperledger is an ecosystem of different tools, libraries and frameworks for creating different types of Blockchains: from private and permissioned to public and permissionless. Fabric is one of those Blockchain frameworks. It allows multiple actors to share a blockchain between themselves for any Use Case which to they could think to apply it.\\
	
	\subsection{Project Motivation}
		\hspace{10mm}This project attempts to explore use of this technology in the security space, namely having a dedicated security network between actors. These actors can be two (or more) organizations looking to join forces and pool security resources. If fact, how this idea got started was thinking about having a dedicated security network as an overlay to something like a Wide-Area Network (WAN) specifically that of a Metropolitan-Area Network (MAN). You could have multiple organizations within a city each helping to strengthen the security mission of their networks without having that security centralized as a city is often its own ecosystem of businesses, departments, and other stakeholders. An extended hope is that this could one day be applied to non-permissioned and/or public blockchains in future work.\\
	
		\hspace{10mm}The choice to start with using Fabric for this idea was the ability to have full control of the blockchain in question while the structure was being planned out and the scripts and chaincode were being developed. A public blockchain such as Etheruem sounded like too many unknown variables right out-of-the-gate. That being said, and as mentioned above, it is a hope that once the plans are solidified translating this work to a public blockchain won't be too difficult. However, something thought about after this choice was made was doubling down on the idea of using a permissioned blockchain such as Fabric for a real implementation, such as for a MAN. The beauty of the idea is that its easily translatable to many different platform. For security network applications, I believe this could have great value in the realms of Public Key Infrastructure (PKI), Two-Factor Authentication (2FA), Distributed Denial of Service (DDoS) prevention, Domain Name System Security (DNSSec), and beyond! The Proof-of-Concept of his research project will be a single blockchain -- or Channel in Hyperledger parlance -- to be used as a distributed log aggregator. As anyone in blue team security can tell you, the logs are everything!\\
	
	\subsection{Project Objectives}
		\hspace{10mm}The objectives of this project are:
		\begin{itemize}
			\item To demystify the Hyperledger Fabric Distributed Ledger Technology
			\item To test the applications of a permissioned blockchain system in the realm of Computer Security
			\item TODO: Add a couple more objectives.
		\end{itemize}
	\subsection{Key Achievements} 
		\hspace{10mm}I am the Key Achievements.\\