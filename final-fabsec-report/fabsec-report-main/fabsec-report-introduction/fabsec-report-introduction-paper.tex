\section{Introduction}
	\hspace{10mm}Blockchain technology is a disruptive force in this world. Typically it is has been world in the world of finance.
	
	\hspace{10mm}FabSec (short for Fabric Security) is an exploration in the potential of using Hyperledger Fabric's Distributed Ledger Technology as a dedicated security network. Hyperledger is an ecosystem of different tools, libraries and frameworks for creating different types of Blockchains: from private and permissioned to public and permissionless. Fabric is one of those Blockchain frameworks. It allows multiple actors to share a blockchain between themsevles for any Use Case which to they could think to apply it.
	
	\hspace{10mm}This project attempts to explore use of this technology in the security space, namely having a dedicated security network between actors. These actors can be two (or more) organizations looking to join forces and pool security resources. How this idea started was thinking about having a dedicated security network as an overlay to something like a Wide-Area Network (WAN) specifically that of a Metropolitan-Area Network (MAN). An extended hope is that this could one day be applied to non-permissioned and/or public blockchains in future work.
	
	\hspace{10mm}The choice to start with using Fabric for this idea was the ability to have full control of the blockchain in question while the structure was being planned out and the scripts and chaincode were being developed. A public blockchain such as Etheruem sounded like too many unknown variables right out-of-the-gate. That being aid, and as mentioned above, it is a hope that once the plans are solidifed translating this work to a public blockchain won't be too difficult. For security network applications, I believe this could have great value in the realms of Public Key Infrastructure (PKI), Two-Factor Authentication (2FA), Distributed Denial of Service (DDoS) prevention, Domain Name System Security (DNSSec), and beyond! The Proof-of-Concept of his research project will be a single blockchain -- or Channel in Hyperledger parlance -- to be used as a distributed log aggregator.
	
	\subsection{Project Objectives}
		\hspace{10mm} I an the Project Objectives.
